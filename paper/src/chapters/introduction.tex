\chapter{Introduction}

Commonly, modeling the fractal nature of planet-like features is done by applying a noise
algorithm---e.g., Perlin Noise \citep{perlin83}---to a sphere surface, using \textsc{uv}-coordinates
(and, sometimes, \textsc{uvw}-coordinates, when a three-dimensional noise function is used) as
inputs to the noise function, in effect, applying a texture to the sphere surface in some sense.
The planet-like features can then be visualized in a number of ways; for example, using the noise
texture simply for coloring the surface of the sphere (\textit{texture mapping}).  This, however,
will not provide a very intersting result, since it will not affect the sphere's lighting---an
important factor when it comes to visual realism.  Taking it a step further, the texture can be
applied as a \textit{bump map}.  A bump map will affect the sphere's lighting, but not its physical
surface \citep{blinn78}, limiting the applicable cases to those that do not put high demands on
visual realism, or to model surface details rather than large surface displacements.

Here, an algorithm is proposed for generating planet-like features on a sphere surface by displacing
its geometry iteratively, doing a simple transformation for each iteration.
